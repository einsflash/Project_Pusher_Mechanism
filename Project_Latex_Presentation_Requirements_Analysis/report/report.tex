\documentclass{book}

\usepackage{amsmath}
\usepackage{amssymb}
\usepackage{amsxtra}
\usepackage{graphicx}
\usepackage{listings}

\newcommand{\refchapter}[1]{Chapter~\ref{#1}}
\newcommand{\refsec}[1]{Section~\ref{#1}}
\newcommand{\refeqn}[1]{Equation~(\ref{#1})}
\newcommand{\reffig}[1]{Figure~\ref{#1}}

\title{\bf Software Lab \\ Computational Engineering Science \\
{\small Report (Template)}} 

\author{Uwe Naumann\footnote{Informatik 12: Software and Tools for Computational Engineering, RWTH Aachen University, {\tt info@stce.rwth-aachen.de}}}
\date{\includegraphics[width=.6\textwidth]{rwth_i12_softw-werkz_en_rgb}}

\begin{document}

\lstloadlanguages{[ISO]C++}
\lstset{basicstyle=\small, numbers=left, numberstyle=\footnotesize,
  stepnumber=1, numbersep=5pt, breaklines=true, escapeinside={/*@}{@*/}}

\pagestyle{headings}

\maketitle

\tableofcontents

\chapter*{Preface}

\begin{itemize}
\item administrative information about the project (e.g, topic issued by which institute)
\item fit of topic into study program (e.g, sufficient prior knowledge)
\item acknowledgement of supervision
\end{itemize}

\chapter{Analysis} \label{ch:analysis}

\section{User Requirements}

user requirements explained (includes essential information and references into literature on technical background of the topic, e.g, \cite{Ries1522Rad}) based on UML Use Case diagram(s)

\section{System Requirements}

functional and non-functional system requirements explained

\chapter{Design} \label{ch:design}

\section{Principal Components and Third-Party Software}

libraries that you built on explained briefly and references to further information

\section{Class Models}

UML Class diagram(s) and description; should link into overall design through
reference of application programming interfaces (API) of third-party software

\chapter{Implementation} \label{ch:implementation}

\section{Development Infrastructure}

programming language, compiler, run time libraries, target platform
(hardware, operating system)

\section{Source Code}

overview of source code structure (file names, directories); build instructions; references into source code documentation e.g, doxygen\footnote{\tt https://github.com/doxygen/doxygen}; short (!) code listings
\begin{lstlisting}
#include<iostream>
int main() {
  std::cout << "Leave me alone world!" << std::endl;
  return 42;
}
\end{lstlisting}
if helpful (must come with detailed explanation)

\section{Software Tests}

e.g, googletest\footnote{\tt https://github.com/google/googletest}

\chapter{Project Management} \label{ch:projectmanagement}

who did what, when, and why; organization of collaboration, i.e. [online] meetings, software version control (e.g, git\footnote{\tt https://git.rwth-aachen.de}

\bibliographystyle{plain}
\bibliography{literature}

\appendix

\chapter{User Documentation} \label{ch:userdoc}

\section{Building}

e.g, using cmake\footnote{\tt https://cmake.org/} and make\footnote{\tt https://www.gnu.org/software/make/}


\section{Testing}

e.g, \verb!make test!

\section{Running}

documented sample session(s); e.g, \verb!make run!

\end{document}

