\documentclass[ucs,10pt]{beamer}
% Template for talks using the Logo of STCE and Corporate Design of RWTH Aachen
% adapted from:
% https://www.mi.fu-berlin.de/w/Mi/BeamerTemplateCorporateDesign

\usepackage{amsmath,dsfont,listings}

%%% STCE logo
% small version for upper right corner of normal pages
\pgfdeclareimage[height=0.9cm]{university-logo}{rwth_i12_softw-werkz_en_rgb.png}
\logo{\pgfuseimage{university-logo}}
% large version for upper right corner of title page
\pgfdeclareimage[height=1cm]{big-university-logo}{rwth_i12_softw-werkz_en_rgb.png}
\newcommand{\titleimage}[1]{\pgfdeclareimage[height=2cm]{title-image}{#1}}
\titlegraphic{\pgfuseimage{title-image}}
%%% end STCE logo

% NOTE: 1cm = 0.393 in = 28.346 pt;    1 pt = 1/72 in = 0.0352 cm
\setbeamersize{text margin right=3.5mm, text margin left=7.5mm}  % text margin

% colors to be used
\definecolor{text-grey}{RGB}{51, 51, 51} % grey text on white background
\definecolor{bg-grey}{rgb}{0.66, 0.65, 0.60} % grey background (for white text)
\definecolor{rwth-blue}{RGB}{0, 83, 159} % blue text
\definecolor{rwth-green}{RGB}{153, 204, 0} % green text
\definecolor{rwth-red}{RGB}{204, 0, 0} % red text (used by \alert)

% switch off the sidebars
% TODO: loading \useoutertheme{sidebar} (which is maybe wanted) also inserts
%   a sidebar on title page (unwanted), also indents the page title (unwanted?),
%   and duplicates the navigation symbols (unwanted)
\setbeamersize{sidebar width left=0cm, sidebar width right=0mm}
\setbeamertemplate{sidebar right}{}
\setbeamertemplate{sidebar left}{}
%    XOR
% \useoutertheme{sidebar}

% frame title
% is truncated before logo and splits on two lines
% if neccessary (or manually using \\)
\setbeamertemplate{frametitle}{%
    \vskip-30pt \color{purple}\large%
    \begin{minipage}[b][23pt]{80.5mm}%
    \flushleft\insertframetitle%
    \end{minipage}%
}

%%% title page
% TODO: get rid of the navigation symbols on the title page.
%   actually, \frame[plain] *should* remove them...
\setbeamertemplate{title page}{
% upper right: STCE logo
\vskip2pt\hfill\pgfuseimage{big-university-logo} \\
\vskip6pt\hskip3pt
% title image of the presentation
% set the title and the author
\begin{center}
\vskip4pt
\large \inserttitle \vskip5pt  \small \insertsubtitle
\vskip8pt
	\normalsize \insertauthor %\\ 
	%\includegraphics[width=2cm]{../../foto_naumann}	
\\ [5mm]
	\footnotesize \insertinstitute 
\end{center}
}
%%% end title page

%%% colors
\usecolortheme{lily}
\setbeamercolor*{normal text}{fg=black,bg=white}
\setbeamercolor*{alerted text}{fg=rwth-red}
\setbeamercolor*{example text}{fg=rwth-green}
\setbeamercolor*{structure}{fg=rwth-blue}

\setbeamercolor*{block title}{fg=white,bg=black!50}
\setbeamercolor*{block title alerted}{fg=white,bg=black!50}
\setbeamercolor*{block title example}{fg=white,bg=black!50}

\setbeamercolor*{block body}{bg=black!10}
\setbeamercolor*{block body alerted}{bg=black!10}
\setbeamercolor*{block body example}{bg=black!10}

\setbeamercolor{bibliography entry author}{fg=rwth-blue}
% TODO: this doesn't work at all:
\setbeamercolor{bibliography entry journal}{fg=text-grey}

\setbeamercolor{item}{fg=rwth-blue}
\setbeamercolor{navigation symbols}{fg=text-grey,bg=bg-grey}
%%% end colors

%%% headline
\setbeamertemplate{headline}{
\vskip4pt\hfill\insertlogo\hspace{3.5mm} % logo on the right

\vskip6pt\color{rwth-blue}\rule{\textwidth}{0.4pt} % horizontal line
}
%%% end headline

%%% footline
\newcommand{\footlinetext}{\insertshortinstitute, \insertshorttitle}
\setbeamertemplate{footline}{
\vskip5pt\color{rwth-blue}\rule{\textwidth}{0.4pt}\\ % horizontal line
\vskip2pt
\makebox[123mm]{\hspace{7.5mm}
\color{rwth-blue}\footlinetext
\hfill \raisebox{-1pt}{\usebeamertemplate***{navigation symbols}}
\hfill \insertframenumber}
\vskip4pt
}
%%% end footline

%%% settings for listings package
\lstset{extendedchars=true, showstringspaces=false, basicstyle=\footnotesize\sffamily, tabsize=2, breaklines=true, breakindent=10pt, frame=l, columns=fullflexible}
\lstset{language=C++} % this sets the syntax highlighting
\lstset{mathescape=true} % this switches on $...$ substitution in code
% enables UTF-8 in source code:
\lstset{literate={ä}{{\"a}}1 {ö}{{\"o}}1 {ü}{{\"u}}1 {Ä}{{\"A}}1 {Ö}{{\"O}}1 {Ü}{{\"U}}1 {ß}{\ss}1}
%%% end listings
  
\usepackage{multicol}
\usepackage{caption}
\usepackage{subcaption}
\usepackage{moresize}

\begin{document}
\title[{\tt info@stce.rwth-aachen.de}]{\textcolor{rwth-blue}{Software Lab Computational Engineering Science} \vspace{.2cm} \\ {\small Group 12, Pusher Mechanism}}
\author[Group 12, Pusher Mechanism]{Aaron Floerke, Arseniy Kholod, Xinyang Song and Yanliang Zhu} 
\institute[Software Lab CES]{
{Informatik 12: Software and Tools for Computational Engineering (STCE)} \\ RWTH Aachen University \vspace{.5cm}
}
\date[]{}

\begin{frame}[plain]
\titlepage
\end{frame}

\begin{frame}
	\frametitle{Contents}
	\vspace*{2mm}
	\tableofcontents
\end{frame}

\section{Preface}

\begin{frame}
\frametitle{Preface \\
	\small \color{rwth-blue} Four-bar linkage model}
	\begin{center}
		\includegraphics[width=\linewidth]{./Figures/GUI_screen.png}
	\end{center}
\end{frame}

\subsection{Introduction}





%\begin{frame}
%	\frametitle{Introduction \\
	%		\small \color{rwth-blue} Degree of freedom}
%	
%	\begin{itemize}
	%		\item \textbf{Formular:}
	%		\[
	%		M = 3(N - 1 - j) + \sum_{i=1}^{j} f_i
	%		\]
	%		\begin{itemize}
		%			\item \( M \): DOF
		%			\item \( N \): Number of links. (4)
		%			\item \( j \): Number of joints. (4 revolute joints)
		%			\item \( f_i \): DOF provided by each joint \( i \), equal 1 for revolute (rotational) joints.
		%		\end{itemize}
	
	%		\item Calculation for the four-bar linkage:
	%		\[
	%		M = \sum_{i=1}^{j} f_i - 3 = 4(1) - 3 = 1
	%		\]
	%		
	%		\item Four-bar linkage has 1 degree of freedom, meaning the mechanism can be fully controlled by a single input. (We use angular velocity as input)
	%	\end{itemize}

%	{\tiny \url{https://en.wikipedia.org/wiki/Degrees_of_freedom_(mechanics)}}
%\end{frame}

\begin{frame}
\frametitle{Introduction \\
    \small \color{rwth-blue} Grashof's Theorem}

    \begin{center}
        \includegraphics[width=\linewidth]{./Figures/introduction/introduction_classification.png}
    \end{center}

    {\tiny \url{https://www.cs.cmu.edu/~rapidproto/mechanisms/chpt5.html}}
\end{frame}

\section{Analysis}

\subsection{User Requirements}

\begin{frame}
\frametitle{Analysis \\
	\small \color{rwth-blue} User Requirements}
	  \begin{minipage}{\linewidth}
		\centering
		\begin{minipage}{0.6\linewidth}
			\begin{itemize}
				\item Implement 27 motion types of the four-bar linkage with one bar fixed:
				\begin{itemize}
					\item Classification values:
					\begin{itemize}
						\item $T_1 = g + h - b - a$
						\item $T_2 = b + g - h - a$
						\item $T_3 = h + b - g - a$
					\end{itemize}
				\end{itemize}
				\item Implement GUI with motion animation and the ability to choose geometrical parameters:
				\begin{itemize}
					\item Length of the bars
					\item Position of the coupler
					\item Input angle
					\item Angle relative to the horizon
					\item Classification values as alternative input
				\end{itemize}
			\end{itemize}
		\end{minipage}
		\hspace{0.05\linewidth}
		\begin{minipage}{0.31\linewidth}
			\begin{figure}[h]
				\includegraphics[width=\textwidth]{./Figures/motion_classification.pdf}
			\end{figure}
		\end{minipage}
	\end{minipage}
	{\tiny Figure from "Classification, geometrical and kinematic analysis of four-bar linkages" 10.15308/Sinteza-2018-261-266 by Ivana Cvetkovic et al.}
\end{frame}

\begin{frame}
\frametitle{Analysis \\
	\small \color{rwth-blue} User Requirements}
	\begin{center}
		\includegraphics[width=0.75\linewidth]{./Figures/optimization_problem.png}
	\end{center}
	\begin{itemize}
			\item Solve an optimization problem:
			\begin{itemize}
				\item Push box with size $80\times60$ from $x=220$ to $x=0$
				\item Do not cross the area of the labelling machine (Area with $x<80$ and $y>70$).
				\item Pass above points $(120, 80)$ and $(220, 80)$
			\end{itemize}
	\end{itemize}
\end{frame}


\subsection{System Requirements}

\begin{frame}
	\frametitle{System Requirements \\
		\small \color{rwth-blue} Functional}
	\begin{itemize}
		\item \textbf{Four-bar linkage model}:
		\begin{itemize}
			\item System simulates all the motion types of the four-bar linkage.
			\item System does not crash with any input of geometrical configuration.
		\end{itemize}
		\item \textbf{Tests}:
		\begin{itemize}
			\item Implement test cases for geometry.
			\item Implement test cases with bad input to test system stability.
		\end{itemize}
		
		\item \textbf{Graphical User Interface}:
		\begin{itemize}
			\item GUI provides the four-bar linkage visualization and motion animation.
			\item User can input geometrical data by moving a point on a slide bar.
			\item GUI is coupled with the four-bar linkage model to use implemented motion cases for animation.
			\item GUI provides tracing for trajectories of the points.
			\item GUI classifies of the linkage.
		\end{itemize}
		\item \textbf{Optimization problem}:
		\begin{itemize}
			\item It should be possible to find a solution (manually) for the optimization problem using the four-bar linkage model.
			\item GUI visualizes the solution.
		\end{itemize}
	\end{itemize}
\end{frame}

\begin{frame}
	\frametitle{System Requirements \\
		\small \color{rwth-blue} Non-Functional}
	\begin{itemize}
		\item \textbf{Performance}:
		\begin{itemize}
			\item The four-bar linkage model is fast enough to provide smooth GUI animations.
			\item GUI animations are not slower than 30 frames per second.
		\end{itemize}
		\item \textbf{Usability}:
		\begin{itemize}
			\item Every essential part of the four-bar linkage model is well documented.
			\item GUI is easy to operate and all functionalities are self-explanatory.
			\item GUI source code is well documented.
		\end{itemize}
	\end{itemize}
\end{frame}

\section{Design}

\begin{frame}
	\frametitle{Design \\
		\small \color{rwth-blue} Development Infrastructure}
	\begin{itemize}
		\item \textbf{1. Operating System:}
		\begin{itemize}
			\item Xubuntu/Windows
		\end{itemize}
		\item \textbf{2. Developing Environment:}
		\begin{itemize}
			\item Programming Language: Python.
			\item IDE: Spyder/Pycharm.
			\item Package Manager: Anaconda.
		\end{itemize}
		\item \textbf{3. Libraries:}
		\begin{itemize}
			\item Frontend: tkinter, math, numpy
			\item Backend: math, numpy
		\end{itemize}
		\item \textbf{4. Version Control System:}
		\begin{itemize}
			\item GitHub: Remote code repositories for team collaboration, code reviews, and version control. \url{https://github.com/einsflash/Project_Pusher_Mechanism}
		\end{itemize}
		\item \textbf{5. Frameworks:}
		\begin{itemize}
			\item Pdoc: Used for generating project documentation, helping the team understand and maintain the code better.
			\item Makefile: For build management.
		\end{itemize}
	\end{itemize}
\end{frame}

\subsection{Class Model(s)}

\begin{frame}
\frametitle{Design \\
	\small \color{rwth-blue} Class Model(s)}
	\begin{figure}
		\centering
		\begin{subfigure}[b]{0.495\textwidth}
			\centering
			\includegraphics[width=\textwidth]{./Figures/class_model/doc_gui.png}
		\end{subfigure}
		\hfill
		\begin{subfigure}[b]{0.495\textwidth}
			\centering
			\includegraphics[width=\textwidth]{./Figures/class_model/doc_four_bar_linkage.png}
		\end{subfigure}
	\end{figure}
\end{frame}

\section{Implementation}

\subsection{Development Infrastructure}

\subsection{Four-Bar Linkage Model}

\begin{frame}
\frametitle{Implementation \\
	\small \color{rwth-blue} Four-Bar Linkage Model}
\end{frame}

\subsection{Software Tests}

\begin{frame}
    \frametitle{Implementation \\
    \small \color{rwth-blue} Software Tests}
    
    \begin{itemize}
        \item Tests classify motion of four-bar linkage based on values \( T_1 \), \( T_2 \), and \( T_3 \)
        \item Verifies correct behavior for Crank, Rocker, and intermediate states
        \item Covers all possible combinations of positive, negative, and zero values
        \item Ensures accurate classification of input and output links
        \item Test cases include Crank-Rocker, Double Crank, Double Rocker, and Rocker-Crank scenarios
        \item Each test checks specific link motion configurations
        \item Automated with \texttt{unittest} framework for reproducibility and consistency
    \end{itemize}
    
\end{frame}
\subsection{GUI}

\begin{frame}[fragile]
\frametitle{Implementation \\
	\small \color{rwth-blue} GUI, Tkinter Intro}
	\begin{center}
		\vspace*{-1mm}
		\includegraphics[width=0.78\linewidth]{./Figures/Implementation/GUI/tkinter_grid.pdf}
		\begin{itemize}
			\item Initiate all tkinter objects inside GUI class and generate app window:
		\end{itemize}
		\begin{lstlisting}
			GUI().tk.mainloop()
		\end{lstlisting}
	\end{center}
\end{frame}

\begin{frame}[fragile]
\frametitle{Implementation \\
	\small \color{rwth-blue} GUI, Animation}
	\vspace*{-0.7mm}
	\begin{itemize}
		\item Update objects in \lstinline|tk.Canvas| every animation step using \lstinline|coords| and/or \lstinline|itemconfigure| for optimization
	\end{itemize}
	\vspace*{-1.8mm}
	\begin{lstlisting}[basicstyle=\ssmall]
		class GUI:
			def __init__(self):
				...
				self.init_toolbar()
				...
			def init_toolbar(self):
				...
				self.enable_animation = tk.IntVar()
				self.animation_button = tk.Checkbutton(self.toolbar_frame, text="animation", 
				                                      variable=self.enable_animation,
				                                      onvalue=1, offvalue=0, command=self.animation)
				self.animation_button.grid(sticky="W", row=10, column=2)
				...
			def refresh(self):
				...
				self.linkage.run()
				...
				self.update_linkage_display()
			def animation(self):
				self.run_animation()
			def run_animation(self):
				if self.enable_animation.get():
					self.linkage.animation_alpha() # alpha = alpha + d_alpha
					self.refresh()
					self.tk.after(25, self.run_animation)
			def update_linkage_display(self):
				...
				self.model_animation.coords(self.model_animation.AB_line, [A_x, A_y, B_x, B_y])
				...			
	\end{lstlisting}
\end{frame}

\begin{frame}[fragile]
\frametitle{Implementation \\
	\small \color{rwth-blue} GUI, Show and hide objects}
	\begin{itemize}
		\item To display different modes, some objects have to be hidden or shown.
		\item For objects in \lstinline|tk.Canvas| use \lstinline|itemconfigure|:
		\begin{itemize}
			\item Hide: \lstinline|self.model_animation.itemconfigure(self.model_animation.AB_line, state='hidden')|
			\item Show: \lstinline|self.model_animation.itemconfigure(self.model_animation.AB_line, state='normal')|
		\end{itemize}
		\item For widgets like \lstinline|tk.Scale| or \lstinline|tk.Text|:
		\begin{itemize}
			\item Hide: \lstinline|self.slider_T1.grid_remove()|
			\item Show: \lstinline|self.slider_T1.grid()|
		\end{itemize}
	\end{itemize}
\end{frame}

\begin{frame}[fragile]
\frametitle{Implementation \\
	\small \color{rwth-blue} GUI, Invalid Setup Handling}
	\begin{center}
		\includegraphics[width=\linewidth]{./Figures/Implementation/invalid_setup.png}
	\end{center}
\end{frame}

\begin{frame}[fragile]
\frametitle{Implementation \\
	\small \color{rwth-blue} GUI, Invalid Setup Handling}
	\begin{lstlisting}[basicstyle=\ssmall]
		class GUI:
			def __init__(self):
				...
				self.init_linkage_display()
				...
			...
			def init_linkage_display(self):
				self.model_animation.invalid_text = self.model_animation.create_text(round(self.model_animation.width/2),
				                                                              round(self.model_animation.height/2),
				                                                              text="Invalid setup, change geometrical values",
				                                                              fill="black", font=('Helvetica 11 bold'))
				self.model_animation.itemconfigure(self.model_animation.invalid_text, state='hidden')
				...
			...
			def update_linkage_display(self):
				if self.linkage.geometric_Validity:
					self.show_linkage()
					if self.enable_optimization_problem.get():
						self.show_optimization_problem()
					self.model_animation.itemconfigure(self.model_animation.invalid_text, state='hidden')
				else:
					self.hide_linkage()
					self.hide_optimization_problem()
					self.model_animation.itemconfigure(self.model_animation.invalid_text, state='normal')
					return
				...
	\end{lstlisting}
\end{frame}

\section{Results}

\subsection{27 movement types}

\begin{frame}
\frametitle{Results \\
	\small \color{rwth-blue} 27 movement types}	
	\begin{center}
		\begin{tabular}{ c@{\hskip 5pt}c@{\hskip 5pt}c@{\hskip 5pt}c@{\hskip 5pt}c}
			\begin{minipage}{0.185\linewidth}\begin{center} \includegraphics[width=\linewidth]{./Figures/27_motion_cases/111.png} \hfill {\tiny $T_{1,2,3} = 1.0, 1.0, 1.0$}\end{center}\end{minipage}& \begin{minipage}{0.185\linewidth}\begin{center} \includegraphics[width=0.55\linewidth]{./Figures/27_motion_cases/011.png} \hfill {\tiny $T_{1,2,3} = 0.0, 1.0, 1.0$}\end{center}\end{minipage}& \begin{minipage}{0.185\linewidth}\begin{center} \includegraphics[width=0.55\linewidth]{./Figures/27_motion_cases/-111.png} \hfill {\tiny $T_{1,2,3} = -1.0, 1.0, 1.0$}\end{center}\end{minipage}& \begin{minipage}{0.185\linewidth}\begin{center} \includegraphics[width=\linewidth]{./Figures/27_motion_cases/101.png} \hfill {\tiny $T_{1,2,3} = 1.0, 0.0, 1.0$}\end{center}\end{minipage}& \begin{minipage}{0.185\linewidth}\begin{center} \includegraphics[width=0.9\linewidth]{./Figures/27_motion_cases/001.png} \hfill {\tiny $T_{1,2,3} = 0.0, 0.0, 1.0$}\end{center}\end{minipage} \\
			\begin{minipage}{0.185\linewidth}\begin{center} \includegraphics[width=0.8\linewidth]{./Figures/27_motion_cases/-101.png} \hfill {\tiny $T_{1,2,3} = -1.0, 0.0, 1.0$}\end{center}\end{minipage}& \begin{minipage}{0.185\linewidth}\begin{center} \includegraphics[width=0.9\linewidth]{./Figures/27_motion_cases/1-11.png} \hfill {\tiny $T_{1,2,3} = 1.0, -1.0, 1.0$}\end{center}\end{minipage}& \begin{minipage}{0.185\linewidth}\begin{center} \includegraphics[width=0.9\linewidth]{./Figures/27_motion_cases/0-11.png} \hfill {\tiny $T_{1,2,3} = 0.0, -1.0, 1.0$}\end{center}\end{minipage}& \begin{minipage}{0.2\linewidth}\begin{center} \includegraphics[width=0.75\linewidth]{./Figures/27_motion_cases/-1-11.png} \hfill {\tiny $T_{1,2,3} = -1.0, -1.0, 1.0$}\end{center}\end{minipage}& \begin{minipage}{0.185\linewidth}\begin{center} \includegraphics[width=0.75\linewidth]{./Figures/27_motion_cases/110.png} \hfill {\tiny $T_{1,2,3} = 1.0, 1.0, 0.0$}\end{center}\end{minipage} \\
			\begin{minipage}{0.185\linewidth}\begin{center} \includegraphics[width=0.65\linewidth]{./Figures/27_motion_cases/010.png} \hfill {\tiny $T_{1,2,3} = 0.0, 1.0, 0.0$}\end{center}\end{minipage}& \begin{minipage}{0.185\linewidth}\begin{center} \includegraphics[width=0.55\linewidth]{./Figures/27_motion_cases/-110.png} \hfill {\tiny $T_{1,2,3} = -1.0, 1.0, 0.0$}\end{center}\end{minipage}& \begin{minipage}{0.185\linewidth}\begin{center} \includegraphics[width=\linewidth]{./Figures/27_motion_cases/100.png} \hfill {\tiny $T_{1,2,3} = 1.0, 0.0, 0.0$}\end{center}\end{minipage}& \begin{minipage}{0.185\linewidth}\begin{center} \includegraphics[width=0.9\linewidth]{./Figures/27_motion_cases/000.png} \hfill {\tiny $T_{1,2,3} = 0.0, 0.0, 0.0$}\end{center}\end{minipage}& \begin{minipage}{0.185\linewidth}\begin{center} \includegraphics[width=0.8\linewidth]{./Figures/27_motion_cases/-100.png} \hfill {\tiny $T_{1,2,3} = -1.0, 0.0, 0.0$}\end{center}\end{minipage} \\
		\end{tabular}
	\end{center}
\end{frame}

\begin{frame}
\frametitle{Results \\
	\small \color{rwth-blue} 27 movement types}	
\begin{center}
	\begin{tabular}{ c@{\hskip 5pt}c@{\hskip 5pt}c@{\hskip 5pt}c}
		\begin{minipage}{0.185\linewidth}\begin{center} \includegraphics[width=0.9\linewidth]{./Figures/27_motion_cases/1-10.png} \hfill {\tiny $T_{1,2,3} = 1.0, -1.0, 0.0$}\end{center}\end{minipage}& \begin{minipage}{0.185\linewidth}\begin{center} \includegraphics[width=0.85\linewidth]{./Figures/27_motion_cases/0-10.png} \hfill {\tiny $T_{1,2,3} = 0.0, -1.0, 0.0$}\end{center}\end{minipage}& \begin{minipage}{0.21\linewidth}\begin{center} \includegraphics[width=0.7\linewidth]{./Figures/27_motion_cases/-1-10.png} \hfill {\tiny $T_{1,2,3} = -1.0, -1.0, 0.0$}\end{center}\end{minipage}& \begin{minipage}{0.185\linewidth}\begin{center} \includegraphics[width=0.7\linewidth]{./Figures/27_motion_cases/11-1.png} \hfill {\tiny $T_{1,2,3} = 1.0, 1.0, -1.0$}\end{center}\end{minipage} \\ \begin{minipage}{0.185\linewidth}\begin{center} \includegraphics[width=0.65\linewidth]{./Figures/27_motion_cases/01-1.png} \hfill {\tiny $T_{1,2,3} = 0.0, 1.0, -1.0$}\end{center}\end{minipage}&
		\begin{minipage}{0.21\linewidth}\begin{center} \includegraphics[width=0.75\linewidth]{./Figures/27_motion_cases/-11-1.png} \hfill {\tiny $T_{1,2,3} = -1.0, 1.0, -1.0$}\end{center}\end{minipage}& \begin{minipage}{0.185\linewidth}\begin{center} \includegraphics[width=0.85\linewidth]{./Figures/27_motion_cases/10-1.png} \hfill {\tiny $T_{1,2,3} = 1.0, 0.0, -1.0$}\end{center}\end{minipage}& \begin{minipage}{0.185\linewidth}\begin{center} \includegraphics[width=0.7\linewidth]{./Figures/27_motion_cases/00-1.png} \hfill {\tiny $T_{1,2,3} = 0.0, 0.0, -1.0$}\end{center}\end{minipage} \\ \begin{minipage}{0.2\linewidth}\begin{center} \includegraphics[width=0.7\linewidth]{./Figures/27_motion_cases/-10-1.png} \hfill {\tiny $T_{1,2,3} = -1.0, 0.0, -1.0$}\end{center}\end{minipage}& \begin{minipage}{0.21\linewidth}\begin{center} \includegraphics[width=0.9\linewidth]{./Figures/27_motion_cases/1-1-1.png} \hfill {\tiny $T_{1,2,3} = 1.0, -1.0, -1.0$}\end{center}\end{minipage}&
		\begin{minipage}{0.21\linewidth}\begin{center} \includegraphics[width=0.55\linewidth]{./Figures/27_motion_cases/0-1-1.png} \hfill {\tiny $T_{1,2,3} = 0.0, -1.0, -1.0$}\end{center}\end{minipage}& \begin{minipage}{0.23\linewidth}\begin{center} \includegraphics[width=0.5\linewidth]{./Figures/27_motion_cases/-1-1-1.png} \hfill {\tiny $T_{1,2,3} = -1.0, -1.0, -1.0$}\end{center}\end{minipage}\\
	\end{tabular}
\end{center}
\end{frame}

\subsection{Optimization problem}

\begin{frame}
\frametitle{Results \\
	\small \color{rwth-blue} Optimization problem}
	\begin{center}
		\includegraphics[width=0.8\linewidth]{./Figures/optimization_problem_solution.png}
	\end{center}
	\begin{itemize}
		\item 9 degrees of freedom (all lengths in cm):
		\begin{itemize}
			\item Length of four bars: $a = 124.0$, $b = 171.2$, $g = 172.1$, $h = 122.6$.
			\item Coupler position: $P_{pos} = 20.0 \%, P_{offset} = 42.0 \%$ of $h$.
			\item Position of point A: $A_x = 27.0$, $A_y = 66.0$.
			\item Angle of ground bar relative to horizon: $\theta = -70.0^{\circ}$
		\end{itemize}
	\end{itemize}
\end{frame}

\section{Documentation}

\begin{frame}
\frametitle{Documentation for Frontend(GUI)}
	\centering
	%\includegraphics[height=0.65\textwidth]{./Figures/class_model/doc_gui.png}
\end{frame}

\begin{frame}
\frametitle{Documentation for Backend}
	\centering
	%\includegraphics[height=0.65\textwidth]{./Figures/class_model/doc_four_bar_linkage.png}
\end{frame}

\section{Project Management}

\begin{frame}
\frametitle{Project Management \\
	\small \color{rwth-blue} Task}
	\begin{itemize}
			\item \textbf{1.Discuss and Design:}
				\begin{itemize}
					\item weekly discussion in discord.
					\item gathering information / generating ideas for program.
				\end{itemize}
			\item \textbf{2.Frontend:}
				\begin{itemize}
					\item Design of GUI
					\item Implementation
					\item Debug
				\end{itemize}
			\item \textbf{3.Backend:}
				\begin{itemize}
				\item Algorithm for calculating positions and angle extremum
				\item Interface for animation
				\item Two types of input
				\item Display information(Grashof condition, geometric validity)
				\end{itemize}
			\item \textbf{4.Test the motion case:}
			\item \textbf{5.Presentation:}
				\begin{itemize}
				\item Analysis (user requirements)
				\item Frontend
				\item Project management
				\item Backend
				\end{itemize}
			\item *The following page outlines the responsibilities of each person.
	\end{itemize}
\end{frame}

\begin{frame}
\frametitle{Project Management \\
	\small \color{rwth-blue} Gantt Chart}

	\begin{center}
		\includegraphics[width=\textwidth]{./Figures/project management 10-18.png}
	\end{center}
\end{frame}

\begin{frame}
\frametitle{Project Management \\
	\small \color{rwth-blue} Task Assignment}

	\begin{flushleft}
		\includegraphics[height=\textheight,keepaspectratio]{./Figures/project management 10-18.png}
	\end{flushleft}
\end{frame}

\section{Live Software Demo}

\begin{frame}
\frametitle{Live Software Demo}

    \begin{enumerate}
        \item Changing the input of slidebar.
        \item Start the animation.
        \item Test different motion types.
        \item Enable points tracing.
        \item Solve the optimization problem.
    \end{enumerate}

\end{frame}

\section{Summary and Conclusion}

\begin{frame}
\frametitle{Summary and Conclusion}
\end{frame}

\begin{frame}
\frametitle{Literature}
	\begin{itemize}
		\item Cvetkovic, Ivana and Stojicevic, Misa and Popkonstantinović, Branislav and Cvetković, Dragan. (2018). Classification, geometrical and kinematic analysis of four-bar linkages. 261-266. 10.15308/Sinteza-2018-261-266.
	\end{itemize}
\end{frame}

\end{document}
